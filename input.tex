\documentclass{exam}
\usepackage[8pt]{extsizes}
%papersize={⟨width⟩,⟨height⟩} 
\usepackage[utf8]{inputenc}
\usepackage[T1]{fontenc}
\usepackage{amsmath}
\usepackage{amsfonts}
\RequirePackage{multicol}
\usepackage{amssymb}
\usepackage{xhfill}
\usepackage[a4paper, total={14cm, 21.6cm},papersize={14cm,21.6cm}]{geometry}
\pointpoints{punto}{puntos}

\usepackage{graphicx}
%\usepackage[width=10.00cm, height=5.00cm]{geometry}

\setlength{\columnsep}{4pc}
\oddsidemargin -0.8in
\textwidth 12.5cm
\topmargin -1.2 in
%\textheight 12in
%\parindent -0.5em
%\parskip 0ex 
%\pagestyle{empty}
\newtheorem{eje}{Ejercicios}
\renewcommand{\labelenumi}{\arabic{enumi})}
\begin{document}
\begin{center}
{\bf Examen Parcial 1 - Cálculo Integral (1000005-B) - Grupo 11}\\
 21/03/2024
\end{center}
Por favor escriba sus respuestas en una hoja aparte, matemáticamente justificadas y usando la escritura correcta. No olvide marcar su hoja de respuestas con sus nombres completos. Durante el examen, no está permitido el uso de cualquier dispositivo electrónico (teléfono celular, tablet, computador, calculadora, etc.) Duración: 80 min.

Resuelva los puntos 5 y 6 y solo 3 de los primeros 4 puntos.

\begin{questions}

	  	\question [1]
  	Halle la derivada de las siguientes funciones
  	\begin{parts}
  	\part   	$
  	g(x) = \sin^2(x) - \int_0^x \cos(t) \;dt
  	$
  	\part $g(x) = 3 + \int_1^{x^2}\sec(t-1)\; dt$
  	\end{parts} 
  	
  	\question [1]
  	Use el teorema fundamental del cálculo para calcular las siguientes integrales definidas:
  	\begin{parts}
  	\part $\int_{-2}^2 (4y^3 + 2y) \; dy$
  	\part $\int_{-1}^0 (2x - e^x )dx$
  	\end{parts}
  	
  	    \question [1]
    Sea la función
    $$
    f(x) = |x-1|
    $$
    \begin{parts}
    \part Dibuje la función
    \part Calcule el valor medio de la función $f$ para el intervalo  $[-1,1]$, y obtenega el valor de $c\in [-1,1]$ tal que satisface el Teorema del Valor Medio del  Cálculo Integral.
    \end{parts}
    
    \question [1]
    Encuentre el área de la región sombreada:
    
    \question [1]
    La tasa de depreciación $dV/dt$ de una máquina es inversamente proporcional al cuadrado de $(2t+1)$, donde $V$ es el valor de la máquina $t$ años después de que se compró. El valor inicial de la máquina fue de $45000$ dólares, y su valor decreció $13000$ dólares en el primer año. Estime su valor después de 7 años.
		
	\question [1]
	Determine el número de subintervalos $n$ necesarios para aproximar $\displaystyle \int_1^2\frac{1}{6x} \; dx$, con error menor que $10^{-4}$ para los siguientes casos:
	\begin{parts}
	\part Usando regla del trapecio
	%\part Usando regla de Simpson
	\end{parts}
	Ayuda: la acotación del error para la regla del Trapecio está dada por la siguiente fórmula:
	$$
	|E_T| \leq \frac{M(b-a) ^3}{12n^2},%\;\;\; |E_S|\leq \frac{M(b-a)^5}{180n^4} 
	$$
	donde $M$ es una cota superior de $|f''(x)|$. 
	%para el caso de la regla del Trapecio y es una cota superior de $|f^{(4)}(x)|$ para el caso de la regla de Simpson.
	\end{questions}		
\end{document}